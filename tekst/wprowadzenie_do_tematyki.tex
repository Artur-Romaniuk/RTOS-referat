W celu dokładnego zrozumienia wszystkich szczegółów komunikacji międzywątkowej w \gls{rtos} (często nazywanych z ang. ),
jak i problemów związanych z nimi, należy najpierw zaznajomić się z ogólnymi
charakterystykami takich systemów.

W tym rozdziale zostaną przedstawione podstawy działania \gls{rtos}
oraz powiązane z nimi ograniczenia. W dalszej części rozdziału zostaną opisane podstawowe zasady
programowania wielowątkowego, co pomoże zrozumieć jakie problemy można napotkać w trakcie
pracy z skomplikowanymi systemami równoległymi.

\section{Systemy Operacyjne Czasu Rzeczywistego}
\Gls{rtos} znacząco różnią się od standardowych systemów operacyjnych. Są to specjalizowane systemy, które zapewniają wysoką niezawodność,
determinizm i szybką odpowiedź na zdarzenia w czasie rzeczywistym.
Są one stosowane w wielu dziedzinach, takich jak przemysł, medycyna, telekomunikacja, transport czy lotnictwo,
gdzie czas reakcji systemu na zdarzenia jest kluczowy dla zapewnienia bezpieczeństwa i efektywności działań.

W RTOS, czas jest podstawowym elementem funkcjonowania systemu. Systemy te muszą zagwarantować przetwarzanie zadań w ściśle określonych przedziałach czasowych,
zwanych czasami odpowiedzi (ang. response time), oraz przestrzeganie terminów odpowiedzi (ang. deadline).
W zależności od dziedziny zastosowania, RTOS może wymagać różnej dokładności czasowej i charakteryzować się różną tolerancją na naruszenia terminów odpowiedzi.
W systemach sterowania przemysłowego czas odpowiedzi wynosi zazwyczaj kilka milisekund, w medycynie może to być kilka mikrosekund,
a w lotnictwie zaledwie kilkadziesiąt mikrosekund. Często też właśnie specyfika aktualnego urządzenia dyktuje konsekwencje związane z niedotrzymaniem terminów.
Ze względu na stosunek do terminów odpowiedzi dzielimy \gls{rtos} na trzy kategorie:
\begin{itemize}
    \item twarde - takie, dla których przekroczenie terminu odpowiedzi wiąże się z katastrofalnymi skutkami dla systemu
    \item sztywne - takie, dla których przekroczenie terminu odpowiedzi oznacza unieważnienie odpowiedzi, jednak bez katastrofalnych skutków dla systemu
    \item miękkie - takie, dla których przekroczenie terminu odpowiedzi oznacza pogorszenie jakości odpowiedzi, bez katastrofalnych skutków dla systemu.
\end{itemize}

Klasycznym przykładem twardego systemu czasu rzeczywistego jest \Gls{abs}.
Jest to urządzenie, które musi się aktywować w ściśle określonym terminie, w celu skutecznego wspomagania kierowcy w trudnej sytuacji. Łatwo sobie wyobrazić,
że przedwczesna lub spóźniona aktywacja takiego systemu może nieść ze sobą tragiczne konsekwencje.

Kategoria sztywnych systemów, nie jest zawsze wydzielana, jednak pozwala na sprecyzowanie systemów pośrednich, dla których przestrzeganie terminów jest
wyjątkowo istotne, lecz pojedyncze przypadki przepuszczenia terminu nie niosą ze sobą katastrofalnych skutków. Tutaj jako przykład takiego systemu można
przedstawić czujnik kontu nachylenia w samolocie. Posiadanie aktualnych wskazań czujnika jest krytyczne dla działania systemu jak i bezpieczeństwa pasażerów,
lecz przepuszczenie jednego pomiaru nie wiąże się od razu z natychmiastową katastrofą samolotu, ponieważ system jest w stanie przez pewien czas polegać
na poprzednich wskazaniach, do momentu rozwiązania problemu.

Najniższymi konsekwencjami cechują się miękkie systemy, ponieważ najczęściej przekroczenie terminów wiąże się jedynie z degradacją jakości systemu.
Przykładem takiego systemu może być domowy termostat z elektronicznym wyświetlaczem. Gdyby doszło do zawieszenia się systemu na pewien czas,
zadowolenie użytkownika najprawdopodobniej spadnie, jednak nie będzie to miało żadnych poważnych konsekwencji dla systemu.

RTOS składa się z trzech głównych elementów: jądra systemu, usług systemowych oraz aplikacji użytkownika.
Jądro systemu to główny komponent \gls{rtos}, który zapewnia podstawowe funkcjonalności, takie jak zarządzanie wątkami, planowanie, obsługa przerwań czy synchronizacja wątków.
Usługi systemowe to zestaw funkcji, które mogą być wywoływane przez aplikacje użytkownika i które zapewniają dodatkowe funkcjonalności,
takie jak zarządzanie wątkami, synchronizacja, czy obsługa systemu plików.
Aplikacje użytkownika to programy, które wykorzystują funkcjonalności RTOS do realizacji swojego zadania.

\subsection{Zalety stosowania RTOS}
\gls{rtos} umożliwiają programistom łatwe tworzenie aplikacji, które są podzielone na wątki, bez stosowania tradycyjnych systemów operacyjnych.
Jest to kluczowy punkt dla zastosowań, gdzie zasoby systemowe są znacznie ograniczone, tak jak np. w systemach wbudowanych.
Alternatywą do \gls{rtos} jest najczęściej architektura \gls{super loop}.



Wątki to jednostki wykonywania, które są wykonywane równolegle i niezależnie od siebie. RTOS zapewnia zarządzanie wątkami, planowanie ich wykonania oraz ich synchronizację w celu zapewnienia bezkonfliktowego dostępu do współdzielonych zasobów systemowych.

Podsumowując, Systemy Operacyjne Czasu Rzeczywistego to specjalizowane systemy operacyjne, które zapewniają niezawodność, determinizm i szybką