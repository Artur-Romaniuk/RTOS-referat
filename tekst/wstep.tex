W dzisiejszych czasach \gls{rtos} znajdują zastosowanie w wielu dziedzinach,
takich jak przemysł, medycyna czy lotnictwo. Jednym z kluczowych elementów tych systemów jest zapewnienie synchronizacji
i efektywnej komunikacji między różnymi wątkami. W niniejszej pracy naukowej zostaną przedstawione
metody synchronizacji oraz komunikacji międzywątkowej w systemach operacyjnych czasu rzeczywistego.

W celu łatwiejszej prezentacji teoretycznych konceptów, w pracy będą zawarte praktyczne przykłady
z wykorzystaniem systemu operacyjnego FreeRTOS. System ten jest jednym z najczęściej stosowanych systemów
czasu rzeczywistego, co w połączeniu z jego prostotą stanowi idealny wybór do przedstawienia
możliwej implementacji omawianych metod.

Praca rozpocznie się od omówienia pojęcia systemów operacyjnych czasu rzeczywistego oraz ich zastosowań.
Następnie przedstawione zostaną podstawowe problemy związane z synchronizacją i komunikacją międzywątkową
w tego typu systemach oraz ich znaczenie dla poprawnego funkcjonowania systemu.
W dalszej części pracy omówione zostaną poszczególne metody synchronizacji, takie jak semafory, czy muteksy
oraz ich zastosowanie w praktyce. Przedstawione zostaną także metody komunikacji
międzywątkowej, takie jak kolejki komunikatów, czy sygnały.
Praca zakończy się podsumowaniem wniosków z analizy poszczególnych sposobów wydajnego zarządzania
wymianą informacji w programach wielowątkowych.

Celem niniejszej pracy naukowej jest przedstawienie różnych metod
synchronizacji i komunikacji międzywątkowej w systemach operacyjnych czasu rzeczywistego oraz ich
znaczenia dla poprawnego i efektywnego działania systemu.
