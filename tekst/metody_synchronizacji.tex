W tym rozdziale przyjrzymy się szczegółowo metodom wymiany informacji pomiędzy wątkami,
analizując fragmenty kodu korzystającego z systemu FreeRTOS.

\section{Semafory}
W tej części pracy zostaną omówione metody synchronizacji dostępu do zasobów współdzielonych, a mianowicie: semafory binarne, semafory zliczające
oraz muteksy. Jako że są to prymitywy synchronizacyjne o bardzo zbliżonym działaniu, duża waga zostanie położona na różnice w ich zastosowaniach
oraz dobrych praktykach dot. wyboru odpowiedniego prymitywu.

\subsection{Semafor binarny}
Semafor jest to rodzaj prymitywu synchronizacyjnego, który nazwą jak i działaniem nawiązuje do kolejowego semafora. W infrastrukturze kolejowej
semafor służy do kontrolowania ruchu pociągów na określonych odcinkach torów. Tak więc analogicznie w kontekście programowania, jako semafor rozumiemy
narzędzie służące do synchronizowania dostępów do ograniczonego zasobu. Semafor binarny posiada dwa stany logiczne: 1 - semafor jest ''otwarty'' oraz
0 - semafor jest ''zamknięty''. Oznacza to kolejno, że chroniony zasób jest wolny lub zajęty.
Zasadniczo semafor posiada dwie podstawowe operacje:
\begin{itemize}
    \item Weź (ang. Take) - operacja zmniejsza wartość semafora o 1 albo blokuje wątek, jeśli wartość semafora jest już równa 0.
    \item Daj (ang. Give) - operacja zwiększa wartość semafora o 1, jeśli był równy 0, i budzi oczekujące wątki.
\end{itemize}

Konceptualnie semafory binarne służą do przekazywania sygnałów pomiędzy fragmentami programu, np. między wątkami lub przerwaniami.
Korzystając z nich użytkownik powinien mieć przed oczami zachowanie kolejowego semafora, który w określonych warunkach otwiera się
i puszcza pociąg na tor. Taka wizualizacja pozwala programiście na uniknięcie typowych błędów, wynikających z błędnego użytkowanie tej metody synchronizacji.
W szczególności semafor binarny nie powinien być wykorzystywany w celu kontrolowania jednoczesnego dostępu do zasobu,
tzn. semafor nie powinien być dawany i brany przez ten sam wątek. W kolejowym przykładzie oznaczałoby to sytuację kiedy to pociąg sam sygnalizowałby
otwarcie i zamknięcie toru, a nie oczekiwał na zewnętrzny sygnał. Jest to o tyle niebezpieczne, ponieważ może prowadzić do zawiłych i trudnych
do analizowania sytuacji, w których jest wiele wątków sterujących, które mogą jednocześnie dawać i brać semafory. Semafor binarny jako stosunkowo
prosta metoda synchronizacji nie rozróżnia pomiędzy tym kto daje, a kto bierze zasób, co może prowadzić do nieoczywistych sytuacji, niezgodnych z wizją
systemu. W takich celach warto przyjrzeć się innej metodzie przedstawionej w dalszej części rozdziału, tj. muteksie.

\subsubsection{Przykład zastosowania}
Oto krótki fragment kodu prezentujący zastosowanie semafora binarnego w systemie FreeRTOS.
\lstinputlisting[language=C++,
    caption=Przykład zastosowania semafora binarnego w FreeRTOS,
    label={lst:semafor_binarny}]{semafor_binarny.cpp}
W tym przykładzie semafor binarny jest wykorzystywany do synchronizacji dostępu do chronionego zasobu przez dwa wątki.
Wartość początkowa semafora jest ustawiona na ''zajęty'', co oznacza, że żadne z zadań nie ma dostępu do zasobu na początku
działania systemu. Sygnał semafora jest dawany przez zewnętrzne przerwanie, które wybudza zadanie oczekujące na semafor.
Zadanie to uzyskuje dostęp do chronionego zasobu i wykonuje swoją pracę.

\subsection{Semafor zliczający}
W odróżnieniu od semafora binarnego, semafor zliczający posiada licznik przyjmujący wartości nieujemne, zamiast tylko zera lub jedynki
jak to było w przypadku semafora binarnego.
Oznacza to to, że architektura systemu korzystającego z semafora zliczającego może zakładać sytuację, kiedy to więcej niż jeden proces
ma dostęp do zasobu w danym czasie. Możliwe zastosowania to różnego rodzaju infrastruktura sieciowa lub bazodanowa,
która obsługuje pewną skończoną, z góry znaną, ilość jednoczesnych połączeń.

Semafor zliczający posiada te same dwie operacje co semafor binarny, z tą różnicą, że operacja dawania, jest wykonywana do momentu osiągnięcia
pewnej zadanej liczby N, oznaczającej maksymalną liczbę jednoczesnych dostępów do zasobu. Oczywistym jest fakt, że za pomocą tak przedstawionej
definicji, możemy w łatwy sposób pokazać, że semafor binarny to nic innego jak semafor zliczający dla N równego 1. Jest to o tyle ważne spostrzeżenie,
ponieważ większość systemów właśnie w taki sposób implementuje semafory binarne.

Tak samo jak w przypadku semafora binarnego, semafor zliczający nie powinien być wykorzystywany w celu kontrolowania jednoczesnego dostępu do zasobu.
Głównym przeznaczeniem semafora zliczającego jest kontrola dostępu do ograniczonego zasobu, który może być jednocześnie wykorzystywany przez wiele procesów.

Oto przykładowy kod zastosowania semafora zliczającego w FreeRTOS:

\lstinputlisting[language=C++,
    caption=Przykład zastosowania semafora zliczającego w FreeRTOS,
    label={lst:semafor_zliczający}]{semafor_zliczajacy.cpp}
W powyższym przykładzie xSemaphoreCreateCounting tworzy semafor zliczający z wartością początkową 3.
W funkcji criticalFunction przed wykonaniem kodu krytycznego wątek musi uzyskać dostęp do semafora, co oznacza,
że musi być dostępnych co najmniej jedno zliczenie semafora. Po zakończeniu pracy zasób jest zwalniany poprzez wywołanie xSemaphoreGive.

W ten sposób semafor zliczający pozwala na równoczesne wykonanie kodu krytycznego przez maksymalnie 3 wątki,
jednocześnie umożliwiając wykonywanie innych czynności przez pozostałe wątki.

\subsection{Muteks}
Muteks to najprawdopodobniej najczęściej wykorzystywany prymityw synchronizacyjny. Jest to rodzaj semafora binarnego, który pozwala na ograniczenie
ilości procesów, które mogą jednocześnie wykonywać krytyczną sekcję kodu. W szczególności muteks pozwala na ograniczenie dostępu do zasobu
do jednego procesu w danym czasie. W momencie, gdy wątek chce uzyskać dostęp do zasobów, sprawdza, czy muteks jest wolny. Jeśli tak,
to wątek blokuje muteks i uzyskuje dostęp do zasobów. Gdy wątek kończy pracę z zasobami, zwalnia muteks, dzięki czemu inny wątek może
uzyskać dostęp do zasobów. Muteksy są więc bardzo pomocne w zapewnieniu bezpiecznego i chronionego dostępu do współdzielonych zasobów.

W odróżnieniu od semafora binarnego, muteks nie pozwala na wywołanie operacji daj bez poprzedniego wywołania operacji weź.
Oznacza, to że muteks przechowuje informację o aktualnym posiadaczu, co w zależności od implementacji może wiązać się też z
dziedziczeniem priorytetów od aktualnego posiadacza. Muteks jest często preferowany w porównaniu do semafora binarnego,
ponieważ ma kilka zalet w kontroli dostępu do zasobów:
\begin{itemize}
    \item Bezpieczeństw - muteks jest bardziej bezpiecznym rozwiązaniem niż semafor binarny, ponieważ tylko jeden wątek
          może uzyskać dostęp do zasobów za każdym razem. W przypadku semafora binarnego, istnieje ryzyko, że dwa lub więcej
          wątków uzyska przypadkowo dostęp do zasobów w wyniku dania dostępu do semafora zanim poprzedni wątek nie skończył swojej pracy,
          co może prowadzić do nieoczekiwanych i trudnych do zdiagnozowania błędów. Muteks gwarantuje tylko jednoczesny dostęp do zasobów przez jeden wątek.
    \item Wydajność - Muteks jest zazwyczaj bardziej wydajny niż semafor binarny, ponieważ wymaga mniej operacji procesora.
          W przypadku semafora binarnego, każdy wątek może zwiększyć lub zmniejszyć wartość licznika semafora, co wymaga dodatkowych operacji procesora.
          W przypadku muteksu tylko jeden wątek blokuje muteks i uzyskuje dostęp do zasobów, co oznacza, że nie ma potrzeby zwiększania lub
          zmniejszania wartości licznika. W praktyce pozwala to na bardziej zoptymalizowaną implementację tej metody synchronizacji.
\end{itemize}

\lstinputlisting[language=C++,
    caption=Przykład zastosowania muteksów w FreeRTOS,
    label={lst:muteks}]{muteks.cpp}

W powyższym kodzie utworzono muteks za pomocą funkcji xSemaphoreCreateMutex(). Następnie utworzono dwa wątki, każdy z których próbuje uzyskać
dostęp do chronionego zasobu za pomocą funkcji xSemaphoreTake(). Po uzyskaniu dostępu do zasobu wątek wykonuje swoje operacje,
a następnie zwalnia muteks za pomocą funkcji xSemaphoreGive(). Funkcja vTaskDelay() została użyta do symulowania pracy wątków.

Powyższy kod przedstawia prosty przykład użycia muteksów w FreeRTOS. Muteksy są przydatnym narzędziem do synchronizacji dostępu do zasobów w systemach
czasu rzeczywistego, ponieważ zapewniają bezpieczne i wydajne zarządzanie dostępem do zasobów przez różne wątki.
