W niniejszej pracy naukowej przedstawiono różne metody synchronizacji i komunikacji międzywątkowej w systemach operacyjnych czasu rzeczywistego.
Wiele systemów operacyjnych czasu rzeczywistego, w tym FreeRTOS, oferuje szereg mechanizmów synchronizacji i komunikacji, które pozwalają
programistom na budowanie aplikacji czasu rzeczywistego z zachowaniem poprawnej synchronizacji i wydajności.

W pracy omówiono podstawowe mechanizmy synchronizacji, takie jak semafory, muteksy, grupy flag oraz kolejki. Każdy z tych mechanizmów ma swoje własne
zastosowania i ograniczenia. Dla przykładu, muteksy są powszechnie stosowane do gwarantowania dostępu do współdzielonych zasobów przez jeden wątek na raz,
natomiast flagi są bardziej odpowiednie dla zadań wymagających przekazywania informacji o stanie systemu między zadaniami.

Podsumowując, w pracy naukowej przedstawiono różne metody synchronizacji i komunikacji międzywątkowej w systemach operacyjnych czasu rzeczywistego.
Dzięki wykorzystaniu tych mechanizmów, programiści mogą budować aplikacje czasu rzeczywistego, które są wydajne, niezawodne i poprawnie synchronizowane.