% !TEX program = xelatex
% !TeX encoding = utf8
% !TeX spellcheck = pl-PL

%%%%%%%%%%%%%%%%%%%%%%%%%%%%%%%%%%%%%%%%%%%%%%%%%%%%%%%%%%%%%%%%%%%%%%%%%%%
% Wybierz rodzaj pracy dyplomowej oraz wydział
% Pick thesis type and faculty
%%%%%%%%%%%%%%%%%%%%%%%%%%%%%%%%%%%%%%%%%%%%%%%%%%%%%%%%%%%%%%%%%%%%%%%%%%%
\documentclass[thesis=inz,faculty=ee]{EE-dyplom} 

% thesis=[inz|mgr|bsc|msc]
%  * inz - praca inżynierska
%  * mgr - praca magisterska
%  * bsc - bachelor thesis
%  * msc - master thesis

% Skróty nazw wydziałów zgodne z domenami internetowymi
% Abbreviations of Faculties according to Internet subdomains
% faculty=[
%	arch,
%	gik,
%	ee,
%	wip
%	]

%%%%%%%%%%%%%%%%%%%%%%%%%%%%%%%%%%%%%%%%%%%%%%%%%%%%%%%%%%%%%%%%%%%%%%%%%%%
% Konfiguracja - do personalizacji
% Configuration - to be personalized
%%%%%%%%%%%%%%%%%%%%%%%%%%%%%%%%%%%%%%%%%%%%%%%%%%%%%%%%%%%%%%%%%%%%%%%%%%%
\instytut{}
\kierunek{ Informatyka Stosowana}
\specjalnosc{Inżynieria Oprogramowania}
\title{Metody komunikacji międzywątkowej w systemach czasu rzeczywistego}
\engtitle{Methods of interthread communication in real time operating systems}
\album{311394}
\author{Artur Romaniuk}
\promotor{}
\date{2023}
\longdate{2023-03-19}

%\grantlicense{TRUE} % [TRUE|FALSE]

%%%%%%%%%%%%%%%%%%%%%%%%%%%%%%%%%%%%%%%%%%%%%%%%%%%%%%%%%%%%%%%%%%%%%%%%%%%
% Streszczenie pracy i abstract.
% In case of thesis in English swap the order - English version goes first.
%%%%%%%%%%%%%%%%%%%%%%%%%%%%%%%%%%%%%%%%%%%%%%%%%%%%%%%%%%%%%%%%%%%%%%%%%%%
\streszczeniepracy{
Celem autora w niniejszej pracy było przedstawienie dostępnych sposobów komunikacji miedzy-wątkowej w systemach operacyjnych czasu rzeczywistego.
W tym celu autor wykorzystał system operacyjny FreeRTOS, na którego to przykładzie zostały przedstawione koncepty synchronizacji zadań oraz przekazywania danych.
Zostały omówione metody synchronizacji, takie jak: semafory, mutexy, grupy flag, kolejki oraz bufory strumieniowe i wiadomości. Autor zwrócił szczególną uwagę
na doktrynalne wykorzystanie prymitywów synchronizacji w systemach czasu rzeczywistego. W szczególności zostały wyszczególnione sytuacje, w których
konkretne metody synchronizacji powinny być preferowane.
} % koniec streszczenia

\slowakluczowe{Systemy operacyjne czasu rzeczywistego, RTOS, FreeRTOS, synchronizacja, wielowątkowość, prymitywy synchronizacyjne}

\thesisabstract{
TBD
} % end of abstract

\thesiskeywords{X, Y, Z}

%%%%%%%%%%%%%%%%%%%%%%%%%%%%%%%%%%%%%%%%%%%%%%%%%%%%%%%%%%%%%%%%%%%%%%%%%%%
% Tu zaczyna się dokument
% Here is the beginning of the document
%%%%%%%%%%%%%%%%%%%%%%%%%%%%%%%%%%%%%%%%%%%%%%%%%%%%%%%%%%%%%%%%%%%%%%%%%%%
\begin{document}
% Strony nagłówkowe
% Headers
\frontpages

% Właściwa treść jest w pliku tekst/main.tex
% Real contents is in tekst/main.tex
% Rozdziały zaczynają się od "chapter"
\chapter{Wstęp}
\label{ch:wstep}
W dzisiejszych czasach \gls{rtos} znajdują zastosowanie w wielu dziedzinach,
takich jak przemysł, medycyna czy lotnictwo. Jednym z kluczowych elementów tych systemów jest zapewnienie synchronizacji
i efektywnej komunikacji między różnymi wątkami. W niniejszej pracy naukowej zostaną przedstawione
metody synchronizacji oraz komunikacji międzywątkowej w systemach operacyjnych czasu rzeczywistego.

W celu łatwiejszej prezentacji teoretycznych konceptów, w pracy będą zawarte praktyczne przykłady
z wykorzystaniem systemu operacyjnego FreeRTOS. System ten jest jednym z najczęściej stosowanych systemów
czasu rzeczywistego, co w połączeniu z jego prostotą stanowi idealny wybór do przedstawienia
możliwej implementacji omawianych metod.

Praca rozpocznie się od omówienia pojęcia systemów operacyjnych czasu rzeczywistego oraz ich zastosowań.
Następnie przedstawione zostaną podstawowe problemy związane z synchronizacją i komunikacją międzywątkową
w tego typu systemach oraz ich znaczenie dla poprawnego funkcjonowania systemu.
W dalszej części pracy omówione zostaną poszczególne metody synchronizacji, takie jak semafory, czy muteksy
oraz ich zastosowanie w praktyce. Przedstawione zostaną także metody komunikacji
międzywątkowej, takie jak kolejki komunikatów, czy sygnały.
Praca zakończy się podsumowaniem wniosków z analizy poszczególnych sposobów wydajnego zarządzania
wymianą informacji w programach wielowątkowych.

Celem niniejszej pracy naukowej jest przedstawienie różnych metod
synchronizacji i komunikacji międzywątkowej w systemach operacyjnych czasu rzeczywistego oraz ich
znaczenia dla poprawnego i efektywnego działania systemu.


\chapter{Wprowadzenie do tematyki}
\label{ch:wprowadzenie_do_tematyki}
W celu dokładnego zrozumienia wszystkich szczegółów komunikacji międzywątkowej w \gls{rtos},
jak i problemów związanych z nimi, należy najpierw zaznajomić się z ogólnymi
charakterystykami takich systemów.

W tym rozdziale zostaną przedstawione podstawy działania \gls{rtos}
oraz powiązane z nimi ograniczenia. W dalszej części rozdziału zostaną opisane podstawowe zasady
programowania wielowątkowego, co pomoże zrozumieć jakie problemy można napotkać w trakcie
pracy z skomplikowanymi systemami równoległymi.

\section{Systemy Operacyjne Czasu Rzeczywistego}
\Gls{rtos} znacząco różnią się od standardowych systemów operacyjnych. Są to specjalizowane systemy, które zapewniają wysoką niezawodność,
determinizm i szybką odpowiedź na zdarzenia w czasie rzeczywistym.
Są one stosowane w wielu dziedzinach, takich jak przemysł, medycyna, telekomunikacja, transport czy lotnictwo,
gdzie czas reakcji systemu na zdarzenia jest kluczowy dla zapewnienia bezpieczeństwa i efektywności działań.

W RTOS, czas jest podstawowym elementem funkcjonowania systemu. Systemy te muszą zagwarantować przetwarzanie zadań w ściśle określonych przedziałach czasowych,
zwanych czasami odpowiedzi (ang. response time), oraz przestrzeganie terminów odpowiedzi (ang. deadline).
W zależności od dziedziny zastosowania, RTOS może wymagać różnej dokładności czasowej i charakteryzować się różną tolerancją na naruszenia terminów odpowiedzi.
W systemach sterowania przemysłowego czas odpowiedzi wynosi zazwyczaj kilka milisekund, a w lotnictwie zaledwie kilkadziesiąt mikrosekund. Często też właśnie specyfika aktualnego urządzenia dyktuje konsekwencje związane z niedotrzymaniem terminów.
Ze względu na stosunek do terminów odpowiedzi dzielimy \gls{rtos} na trzy kategorie:
\begin{itemize}
      \item twarde - takie, dla których przekroczenie terminu odpowiedzi wiąże się z katastrofalnymi skutkami dla systemu
      \item sztywne - takie, dla których przekroczenie terminu odpowiedzi oznacza unieważnienie odpowiedzi, jednak bez katastrofalnych skutków dla systemu
      \item miękkie - takie, dla których przekroczenie terminu odpowiedzi oznacza pogorszenie jakości odpowiedzi.
\end{itemize}

Klasycznym przykładem twardego systemu czasu rzeczywistego jest \gls{abs}.
Jest to urządzenie, które musi się aktywować w ściśle określonym terminie, w celu skutecznego wspomagania kierowcy w trudnej sytuacji. Łatwo sobie wyobrazić,
że przedwczesna lub spóźniona aktywacja takiego systemu może nieść ze sobą tragiczne konsekwencje.

Kategoria sztywnych systemów, nie jest zawsze wydzielana, jednak pozwala na sprecyzowanie systemów pośrednich, dla których przestrzeganie terminów jest
wyjątkowo istotne, lecz pojedyncze przypadki przepuszczenia terminu nie niosą ze sobą katastrofalnych skutków. Tutaj jako przykład takiego systemu można
przedstawić czujnik kątu nachylenia w samolocie. Posiadanie aktualnych wskazań czujnika jest krytyczne dla działania systemu jak i bezpieczeństwa pasażerów,
lecz przepuszczenie jednego pomiaru nie wiąże się od razu z natychmiastową katastrofą samolotu, ponieważ system jest w stanie przez pewien czas polegać
na poprzednich wskazaniach, do momentu rozwiązania problemu.

Najniższymi konsekwencjami cechują się miękkie systemy, ponieważ najczęściej przekroczenie terminów wiąże się jedynie z degradacją jakości systemu.
Przykładem takiego systemu może być domowy termostat z elektronicznym wyświetlaczem. Gdyby doszło do zawieszenia się systemu na pewien czas,
zadowolenie użytkownika najprawdopodobniej spadnie, jednak nie będzie to miało żadnych poważnych konsekwencji dla systemu.

\subsection{Budowa RTOS}
RTOS składa się z trzech głównych elementów: jądra systemu, usług systemowych oraz aplikacji użytkownika.
Jądro systemu to główny komponent \gls{rtos}, który zapewnia podstawowe funkcjonalności, takie jak wątki, planowanie czy obsługa przerwań.
Usługi systemowe to zestaw funkcji, które mogą być wywoływane przez aplikacje użytkownika i które zapewniają interfejs do kontroli działania systemu,
t.j. zarządzanie wątkami, synchronizacja, czy obsługa systemu plików.
Aplikacje użytkownika to programy, które wykorzystują funkcjonalności RTOS do realizacji swojego zadania.

Samo jądro systemu jest najczęściej stosunkowo prostym programem, ponieważ jako centralny moduł zarządzający zasobami systemu musi ono być
wyjątkowo wydajne. Jakikolwiek zbędny narzut obliczeniowy byłby w znacznym stopniu zauważalny na wyższych szczeblach systemu.
Przy tworzeniu jądra systemu, najważniejszym elementem jest zawsze \gls{planista}, tzn. moduł systemu odpowiedzialny za kolejkowanie dostępów
do zasobów systemowych. Niestety zadanie planowania wcale nie należy do trywialnych.

Złożoność problemu planowania polega na tym, że jest to problem obliczeniowo trudny.
W przypadku planowania kolejności wykonywania zadań, musimy przydzielić różne zasoby systemowe (np. czas procesora, pamięć, urządzenia wejścia/wyjścia)
dla wielu zadań, które konkurują o te zasoby. Trudność polega na tym, że musimy znaleźć optymalny przydział zasobów dla każdego zadania,
uwzględniając ich priorytety i wymagania czasowe.
W przypadku systemów czasu rzeczywistego, dodatkowo musimy zagwarantować, że zadania będą wykonywane w określonych ramach czasowych.
W związku z tym, planowanie w systemach czasu rzeczywistego wymaga ciągłej oceny i przewidywania przyszłych wymagań systemu,
a także zapewnienia, że zadania będą wykonywane w odpowiednim czasie, aby uniknąć zagłodzenia lub przeciążenia systemu.

Istnieje wiele algorytmów planowania, które zostały opracowane, aby rozwiązać ten problem, jednak wciąż istnieją ograniczenia,
takie jak złożoność obliczeniowa i trudność w określeniu optymalnego przydziału zasobów. Istnieją również niedeterministyczne algorytmy planowania,
działające na heurystycznych metodach analizy kodu, jednak omówienie tych znacznie bardziej zaawansowanych metod wykracza poza zakres tej pracy.

Najbardziej popularne algorytmy planowania to:
\begin{itemize}
      \item Rate Monotonic Scheduling (RMS) - jest to metoda planowania polegająca na przydziale priorytetów w oparciu o okres zadania,
            gdzie zadania z krótszymi okresami otrzymują wyższe priorytety. Ta metoda jest prosta i efektywna, ale może prowadzić do zagłodzenia
            zadań o dłuższych okresach.
      \item Round Robin (RR) - W metodzie RR, każde zadanie otrzymuje określony kwant czasu (np. 10ms) na przetwarzanie,
            a po upływie tego czasu procesor jest przekazywany do kolejnego zadania, niezależnie od tego, czy bieżące zadanie zostało zakończone.
      \item Earliest Deadline First Scheduling (EDF) - jest to metoda planowania, która przydziela zadaniom priorytety w oparciu o ich terminy wykonywania.
            Zadanie z najkrótszym pozostałym terminem otrzymuje najwyższy priorytet. Ta metoda zapewnia minimalny czas oczekiwania dla każdego zadania,
            ale może prowadzić do trudności z wykonywaniem zadań o wysokiej złożoności obliczeniowej.
      \item Priority Scheduling (PS) - jest to metoda planowania, w której każde zadanie otrzymuje priorytet.
            Zadanie z wyższym priorytetem ma pierwszeństwo przed zadaniem z niższym priorytetem. Ta metoda jest łatwa do zrozumienia i wdrożenia,
            ale może prowadzić do zagłodzenia zadań o niższych priorytetach.
\end{itemize}

Ostatecznie decyzja dot. wybrania konkretnego algorytmu planowania jest jedną z kluczowych cech danego systemu \gls{rtos}.
Przedstawiany w dalszych częściach pracy system FreeRTOS korzysta z priorytetów zadań, co jest naturalnym wyborem dla
systemów wbudowanych, ponieważ klasyczna metoda obsługi przerwań oraz zdarzeń w takich systemach jest właśnie oparta na priorytetach.


\subsection{Zalety stosowania RTOS}
Podstawową zaletą \gls{rtos} jest to, że umożliwiają programistom łatwe tworzenie aplikacji, które są podzielone na wątki,
bez stosowania tradycyjnych systemów operacyjnych i związanego z nimi niedeterminizmu.
Jako że stosowanie zwykłych systemów operacyjnych jest niedopuszczalne w rozwiązaniach o wysokich wymaganiach niezawodności,
jedyną alternatywą do \gls{rtos} jest architektura \gls{super loop}, która obarcza programistę całą logiką podziału zadań.
Nic więc dziwnego, że \gls{rtos} są jednym z najbardziej popularnych oprogramowań stosowanych w systemach krytycznych,
gdzie gwarancje dot. niezawodności, determinizmu i czasu reakcji są wysoko cenione.

\section{Programowanie wielowątkowe}
Programowanie wielowątkowe jest obecnie jednym z najczęściej stosowanych podejść do równoległego programowania
w systemach operacyjnych i aplikacjach. Jednak, z uwagi na złożoność problemu, programowanie wielowątkowe może prowadzić do wielu trudności,
które mogą wpłynąć na wydajność aplikacji. W tym rozdziale zostaną przedstawione podstawowe koncepcje programowania wielowątkowego,
a także omówione najczęstsze trudności i wyzwania, które mogą wystąpić podczas implementacji tego rodzaju aplikacji.

\subsection{Cechy programowania wielowątkowego}
Programowanie wielowątkowe jest techniką programowania, która polega na równoległym wykonywaniu wielu wątków w obrębie jednego procesu.
Wątki to niezależne sekwencje instrukcji, które wykonują określone zadania w ramach spójnej przestrzeni pamięciowej.
Wątki mogą działać równocześnie, co umożliwia wykorzystanie wielu rdzeni procesora oraz zwiększenie wydajności aplikacji,
jednak mogą również służyć wyłącznie do rozdzielenia wykonywanego kodu pod kątem logicznym w celu podziału złożonego zadania na prostsze
części składowe.

Pomimo korzyści wynikających z programowania wielowątkowego, istnieją pewne trudności i wyzwania,
które mogą negatywnie wpłynąć na jakość aplikacji. Wśród najważniejszych trudności należy wymienić:
\begin{itemize}
      \item Zagłodzenie zasobów - sytuacja, w której jeden lub więcej wątków nie może uzyskać dostępu do wymaganych zasobów systemowych,
            co prowadzi do zablokowania lub zawieszenia aplikacji. Zagłodzenie zasobów może być spowodowane zbyt dużą liczbą wątków,
            niewłaściwym planowaniem wątków lub błędami w synchronizacji.
      \item Współdzielenie danych - wspólny dostęp do danych może prowadzić do błędów i nieprzewidywalnego zachowania aplikacji.
            Aby zapobiec tym problemom, należy odpowiednio synchronizować dostęp do pamięci współdzielonej, stosując mechanizmy synchronizacji
            takie jak semafory, muteksy, blokady itp.
      \item Deadlock oraz Livelock - zablokowanie działania systemu w wyniku odpowiednio:
            wzajemnej blokady wątków próbujących uzyskać dostęp do synchronizowanych zasobów,
            niewykonywania pracy w wątkach w wyniku zapętlonego ustępowania sobie wątków.
\end{itemize}

\subsection{Praktyczne aspekty programowania wielowątkowego}
W praktyce w celu uniknięcia wyżej wymienionych problemów, programista musi zwracać szczególną uwagę przy pisaniu
oprogramowania wielowątkowego. Oznacza to minimalizację zbędnego korzystania z pamięci współdzielonej, przemyślane korzystanie
z elementów synchronizacji oraz posiadanie ściśle ustalonych wymagań wobec planisty \gls{rtos}.

Dodatkowo aby stosunek narzutu administracji systemem do właściwej logiki programu był odpowiedni, ilość wątków nie może być nadmiarowa.
W zależności od rodzaju wydzielenia wątków: jednorodne, o zbliżonej złożoności i wymogach obliczeniowych, lub niejednorodne, o różnej złożoności,
system może działać najlepiej w różnych konfiguracjach. Dlatego krytycznym aspektem wytwarzania wydajnego oprogramowania jest testowanie oraz
\gls{benchmarking}, które to metody pozwalają na empiryczne znalezienie optymalnych rozwiązań.

\section{Wątki w systemie FreeRTOS}
Podstawowym elementem systemu \gls{rtos} jest wątek (w nomenklaturze angielskiej nazywany Thread lub Task).
Wątek cechuje się tym, że jest niezależny od pozostałych części programu oraz wykonuje się we własnym kontekście,
z własnym fragmentem stosu.
Wątek nie posiada też żadnych zależności wobec innych zadań oraz jądra systemu. Oznacza to, że wątek może być dowolnie
zatrzymywany oraz startowany, w zależności od decyzji planisty.
Wątek może znajdować się w jednym z następujących stanów:
\begin{itemize}
      \item Aktywny(ang. Running) - wątek jest aktualnie wykonywany oraz ma dostęp do zasobów procesora.
      \item Gotowy(ang. Ready) - wątek jest na liście oczekujących do wykonania, ponieważ wątek o równym lub wyższym
            priorytecie jest aktualnie wykonywany.
      \item Blokowany(ang. Blocked) - wątek oczekuje na zdarzenie czasowe lub zewnętrzne, do którego to momentu wystąpienia
            wątek nie może zostać przeniesiony do stanu aktywnego.
      \item Zawieszony(ang. Suspended) - wątek który celowo został przeniesiony do stanu zawieszenia nie może zostać przeniesiony
            do stanu aktywnego, do momentu celowego odwieszenia wątku za pomocą wywołania odpowiedniej metody.
\end{itemize}

System FreeRTOS jest systemem priorytetowym, co oznacza, że aktualnie aktywny wątek zawsze ma najwyższy lub równy priorytet
w odniesieniu do pozostałych wątków w gotowości. Dodatkowo system może zostać skonfigurowany
aby korzystać z wywłaszczania (ang. preemtion), t.j. zmiana kontekstu może nastąpić bez celowego zatrzymania aktualnie
wykonywanego wątku, a w wyniku zmiany priorytetów lub odblokowania wątku o wyższym priorytecie. \Gls{planista} stosuje algorytm
Round Robin dla zadań o równym priorytecie w celu sprawiedliwego podziału zasobów pomiędzy wątkami o równej sobie ważności.

Oto przykładowy fragment kodu w C++ prezentujący stworzenie oraz następne usunięcie wątku w systemie FreeRTOS.

\lstinputlisting[language=C++,
      caption=Przykład tworzenia oraz usuwanią wątku w FreeRTOS,
      label={lst:task_creation}]{task_creation.cpp}


\chapter{Metody synchronizacji}
\label{ch:metody_synchronizacji}
W tym rozdziale przyjrzymy się szczegółowo metodom wymiany informacji pomiędzy wątkami,
analizując fragmenty kodu korzystającego z systemu FreeRTOS.

\section{Semafory}
W tej części pracy zostaną omówione metody synchronizacji dostępu do zasobów współdzielonych, a mianowicie: semafory binarne, semafory zliczające
oraz muteksy. Jako że są to prymitywy synchronizacyjne o bardzo zbliżonym działaniu, duża waga zostanie położona na różnice w ich zastosowaniach
oraz dobrych praktykach dot. wyboru odpowiedniego prymitywu.
\subsection{Semafor binarny}

\chapter{Referencja}
\input{tekst/reference}

% Bibliografia - musi być
% Bibliography - must exist
\bibliografia

% Strony końcowe - można zakomentować, jeśli zbędne
% Additional pages - comment out if not needed

% Wykaz symboli i skrótów - patrz opis w tekście przykładowym
\acronymslist
% Spis rysunków
% \listoffigures
% Spis tabel
% \listoftables
% Załączniki (plik appendices.tex)
% \easyappendices
\end{document}
%%%%%%%%%%%%%%%%%%%%%%%%%%%%%%%%%%%%%%%%%%%%%%%%%%%%%%%%%%%%%%%%%%%%%%%%%%%
